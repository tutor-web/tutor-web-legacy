\item Finnið lausn/lausnir ef til eru á 2. stigs jöfnunni $$- x^{2} + 9 x + 1=0.$$

a) Jafnan hefur eina lausn og hún er $x=\frac{9}{2}$.

b) $x = - \frac{1}{2} \sqrt{77} + \frac{9}{2}$, og $x=\frac{1}{2} \sqrt{77} + \frac{9}{2}$.

c) $x = - \frac{1}{2} \sqrt{85} + \frac{9}{2}$, og $x=\frac{9}{2} + \frac{1}{2} \sqrt{85}$. % CORRECT

d) Jafnan hefur ekki rauntölulausn


\item Þáttið margliðuna $$f(x) = x^2 -1.$$

a) $f(x) = (x-1)(x+2)$.

b) $f(x) = (x+1)(x-1)$. % CORRECT

c) $f(x) = (x-1)(x-1)$.

d) $f(x) = (x+1)(x+1)$.


\item Gefið er fallið $$\displaystyle\theta(x)=\frac{2x^2-3}{x}.$$ Hvert er gildið á $\theta(-2)$?

a) $\theta(-2)=\frac{5}{2}$

b) $\theta(-2)=-\frac{5}{2}$ % CORRECT

c) $\theta(-2)=\frac{11}{2}$

d) $\theta(-2)=-\frac{11}{2}$


\item Finnið skilgreiningarmengi, $D_f$, fallsins $$f(x) = \frac{1}{\sqrt x}.$$

a) $\mathbb{R} \setminus \{0\}$.

b) $\mathbb{R}_- = \{ x \in \mathbb{R} ; x<0\}$.

c) $\mathbb{R}$.

d) $\mathbb{R}_+ = \{ x \in \mathbb{R} ; x>0\}$. % CORRECT


\item Skilgreinum fleygboga $$y=- 7 x^{2}.$$ Fyrir hvaða gildi á $x$ sker hann $x$-ásinn?

a) Fleygboginn sker $x$ ás einu sinni og það er í $x=0$. % CORRECT

b) Fleygboginn sker ekki $x$ ás

c) Fleygbogin sker $x$ ás einu sinnig og þar er í $x=- \frac{1}{14}$.

d) $x=-7$, og $x=7$.


\item Hver er lengd vigursins $ \mathbf{a} =
\left[
\begin{matrix}
4\\
-3
\end{matrix}
\right]$?

a) 25.

b) 1

c) 7.

d) 5. % CORRECT


\item Finnið hallatölu snertils við fallið $f(x) = \sqrt x
\left( \sqrt{x} + 1\right)$ í punktinum $x=4$.

a) $1/8$.

b) $5/4$. % CORRECT

c) $6$.

d) $3/4$.


\item Þrír menn eiga hver fyrir sig þrjá hatta, svartan, hvítan og gulan. Ef hver þeirra setur upp einhvern af höttum sínum af handahófi og mennirnir hittast síðan, hverjar eru líkurnar á að nákvæmlega tveir þeirra hafi eins hatt?

a) $ \frac{4}{9}$.

b) $ \frac{1}{2}$.

c) $ \frac{2}{3}$. % CORRECT

d) $ \frac{1}{3}$.


\item Látum $f(x)=\sqrt x$ og $h(x)=1/x$. Finnið $h\circ f$.

a) $h \circ f(x) = \sqrt{1}/x$.

b) $h \circ f(x) = \sqrt{\frac 1x}$.

c) $h\circ f(x) = \frac{1}{\sqrt x}$. % CORRECT

d) $h \circ f(x) = 1/x$.


\item Reiknið
$$
11 \cdot 5+2 \cdot  \left ( 3+7 \right ) .
$$
\newpage

a) Ekkert ofangreindra er rétt

b) Svarið er 56

c) Svarið er 75 % CORRECT

d) Svarið er 35


\item Látum $ \mathbf{a} =
\left[
\begin{matrix}
1\\
2
\end{matrix}
\right]
$
og
$
\mathbf{b} =
\left[
\begin{matrix}
2\\
1
\end{matrix}
\right]
$.
Veldu réttan svarmöguleika.

a) $ \mathbf{a} \cdot \mathbf{b} = 1$.

b) Ekkert ofangreint er rétt

c) $ \mathbf{a} - \mathbf{b} = \left[  \begin{matrix} 3 \\ 3 \end{matrix} \right] $.

d) $ \mathbf{a} - \mathbf{b} = \left[  \begin{matrix} -1\\ 1 \end{matrix} \right] $. % CORRECT


\item Diffrið/deildið: $$f(x)=\frac{x+3}{x^2}$$

a) \hspace{2mm} $\displaystyle f'(x)=-\frac{x+6}{x}$

b) \hspace{2mm} $\displaystyle f'(x)=\frac{x-6}{x}$

c) \hspace{2mm} $\displaystyle f'(x)=-\frac{x+6}{x^3}$ % CORRECT

d) \hspace{2mm} $\displaystyle f'(x)=\frac{3x+6}{x^3}$


\item Jón og Guðrún ætla að kasta tening og velta nú fyrir sér nokkrum möguleikum. Þau tákna mengi
allra mögulegra útkoma með
$$
\Omega=\{1,\ 2,\ 3,\ 4,\ 5,\ 6\}.
$$
Þau hafa sérstaklega áhuga á eftirfarandi:
\begin{eqnarray*}
A&=&\left \{ 1,\ 3,\ 5 \right \} \\
B&=&\left \{ 2,\ 4,\ 6  \right \} \\
C&=&\left \{ 1 \right \} \\
D&=&\left \{ 6 \right \}.
\end{eqnarray*}
Slík söfn mögulegra útkoma úr tilraun nefnast atburðir. Um þessa atburði gildir:

a) $B\setminus D=B $

b) $A\cap B=\Omega$

c) $A\setminus B=D$

d) $A\cup B=\Omega$ % CORRECT


\item Hver er hallatala línunnar sem fer í gegnum punktana (-2,1) og (5,6)?

a) \hspace{2mm} $\displaystyle\frac{5}{3}$

b) \hspace{2mm} $\displaystyle\frac{7}{5}$

c) \hspace{2mm} $\displaystyle\frac{1}{3}$

d) \hspace{2mm} $\displaystyle\frac{5}{7}$ % CORRECT


\item Leysið línulega jöfnuhneppið:
\begin{align*}
x + 3y + 2z &= 1, \\
x - 2y + z &= 0, \\
-x - y - z &= 3.
\end{align*}

a) Jöfnuhneppið hefur enga lausn.

b) $x = -8$, $y = -1$ og $z = 6$. % CORRECT

c) $x = 1$, $y = -6$ og $z = 8$

d) $x = -1$, $y = 6$ og $z = -8$


\item Setjum $s_n = \sum_{i=1}^n 1$. Merkið við þann möguleika sem best á við.
\newpage

a) $s_n = n-1$

b) $s_n = n+1$

c) $s_n = 1$

d) $s_n = n$ % CORRECT


\item Einfaldið
$$\log(u^2-2u+1)-\log(u-1)$$

a) $\log(u^2-3u+2)$

b) $\log(u^3-3u^2+3u-1)$

c) $\log(u+1)$

d) $\log(u-1)$ % CORRECT


\item Reiknið
\[
\frac{1}{6} + \frac{2}{3} \left( 2 - \frac{3}{4} \right).
\]

a) 1. % CORRECT

b) $ \frac{11}{12}$.

c) $ \frac{3}{4}$.

d) $ \frac{1}{2}$.


